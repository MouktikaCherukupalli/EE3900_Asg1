\documentclass[journal,11pt,twocolumn]{IEEEtran}
\usepackage{gensymb}
\usepackage{amssymb}
\usepackage[cmex10]{amsmath}
\usepackage{amsthm}
\usepackage[export]{adjustbox}
\usepackage{bm}
\usepackage{longtable}
\usepackage{enumitem}
\usepackage{mathtools}
\usepackage{tikz}
\usepackage[breaklinks=true]{hyperref}
\usepackage{listings}
\usepackage{color}                                            %%
\usepackage{array}                                            %%
\usepackage{longtable}                                        %%
\usepackage{calc}                                             %%
\usepackage{multirow}                                         %%
\usepackage{hhline}                                           %%
\usepackage{ifthen}                                           %%
\usepackage{lscape}    
\usepackage{multicol}
% \usepackage{biblatex}
% \usepackage{enumerate}
\DeclareMathOperator*{\Res}{Res}
\renewcommand\thesection{\arabic{section}}
\renewcommand\thesubsection{\thesection.\arabic{subsection}}
\renewcommand\thesubsubsection{\thesubsection.\arabic{subsubsection}}
\renewcommand\thesectiondis{\arabic{section}}
\renewcommand\thesubsectiondis{\thesectiondis.\arabic{subsection}}
\renewcommand\thesubsubsectiondis{\thesubsectiondis.\arabic{subsubsection}}
\hyphenation{op-tical net-works semi-conduc-tor}
\def\inputGnumericTable{}                                 %%
\lstset{
frame=single, 
breaklines=true,
columns=fullflexible
}
\begin{document}
    \newtheorem{theorem}{Theorem}[section]
    \newtheorem{problem}{Problem}
    \newtheorem{proposition}{Proposition}[section]
    \newtheorem{lemma}{Lemma}[section]
    \newtheorem{corollary}[theorem]{Corollary}
    \newtheorem{example}{Example}[section]
    \newtheorem{definition}[problem]{Definition}
    \newcommand{\BEQA}{\begin{eqnarray}}
    \newcommand{\EEQA}{\end{eqnarray}}
    \newcommand{\define}{\stackrel{\triangle}{=}}
    \newcommand*\circled[1]{\tikz[baseline=(char.base)]{
        \node[shape=circle,draw,inner sep=2pt] (char) {#1};}}
    \bibliographystyle{IEEEtran}
    \providecommand{\mbf}{\mathbf}
    \providecommand{\pr}[1]{\ensuremath{\Pr\left(#1\right)}}
    \providecommand{\qfunc}[1]{\ensuremath{Q\left(#1\right)}}
    \providecommand{\sbrak}[1]{\ensuremath{{}\left[#1\right]}}
    \providecommand{\lsbrak}[1]{\ensuremath{{}\left[#1\right.}}
    \providecommand{\rsbrak}[1]{\ensuremath{{}\left.#1\right]}}
    \providecommand{\brak}[1]{\ensuremath{\left(#1\right)}}
    \providecommand{\lbrak}[1]{\ensuremath{\left(#1\right.}}
    \providecommand{\rbrak}[1]{\ensuremath{\left.#1\right)}}
    \providecommand{\cbrak}[1]{\ensuremath{\left\{#1\right\}}}
    \providecommand{\lcbrak}[1]{\ensuremath{\left\{#1\right.}}
    \providecommand{\rcbrak}[1]{\ensuremath{\left.#1\right\}}}
    \theoremstyle{remark}
    \newtheorem{rem}{Remark}
    
    
\title{EE3900 Assignment1}
\author{ Cherukupalli Sai Malini Mouktika\\\normalsize AI21BTECH11007 \\ \vspace*{10pt} \Large }



\maketitle

\textbf{3.2 : }
Determine the z-transform of the sequence
\begin{equation}
    x[n]= 
    \begin{cases}
    n, &  0 \leq n \leq N-1\\
    N, &  N \leq n
    \end{cases}
\end{equation}
\textbf{Solution :} Z-transform of $x[n]$ is 
\begin{align}
 & \sum_{n=-\infty}^{\infty}x[n]z^{-n}\\
&= \sum_{n=0}^{\infty}x[n]z^{-n}\\
&=  \sum_{n=0}^{N-1}x[n]z^{-n} +  \sum_{n=N}^{\infty}x[n]z^{-n} \\
&=  \sum_{n=0}^{N-1}nz^{-n} +  \sum_{n=N}^{\infty}Nz^{-n}\\
\end{align}
First term is in AGP.
\begin{align}
    \sum_{n=0}^{N-1}nz^{-n} &= z^{-1} + 2z^{-2}+....+ (N-1)z^{-N+1}\\
    &=\frac{z^{-1}-z^{-N}}{(z^{-1}-1)^{2}} - \frac{(N-1)z^{-N}}{1-z^{-1}}\\ 
\end{align}
Second term is in GP.
\begin{align}
    \sum_{n=N}^{\infty}Nz^{-n} &= N(z^{-N}  + z^{-N-1}+.....\\
    &= N\frac{z^{-N}}{1-z^{-1}}
\end{align}
On adding 
\begin{align}
  &=  \frac{z^{-1}-z^{-N}}{(z^{-1}-1)^{2}} - \frac{(N-1)z^{-N}}{1-z^{-1}} +  N\frac{z^{-N}}{1-z^{-1}}\\
  &= \frac{z^{-1}-z^{-N-1}}{(1-z^{-1})^{2}}
\end{align}
Hence Z transform of $x[n]$ is
\begin{align}
\frac{z^{-1}-z^{-N-1}}{(1-z^{-1})^{2}}
\end{align}
\end{document}
